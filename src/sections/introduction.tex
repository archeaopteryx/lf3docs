\section{Physischer Netzwerkplan}\label{sec:Netzwerk}

In diesem Abschnitt werden die Netzwerkverbindungen der am Arbeitsplatz 12 angeschlossenen Computer näher beschrieben, welche als solche mit \computerOne und \computerTwo bezeichnet sind.

\subsection{Alle Netzwerkverbindungen und Netzwerkkomponenten des Arbeitsplatzes}\label{Dokumentation}

An beiden Computern ist die Netzwerkkarte 2 (Network Interface Card/NIC) über ein Cat.5e-Netzwerkkabel mit dem vorderen Patch-Panel am Arbeitsplatz AG-12 verbunden. Ein zweites Cat.5e-Netzwerkkabel leitet weiter zum Hub im Rack des Arbeitsplatzes AG-12. Dieses ist mit dem Ziel-Port im nächsten Rack beschriftet. 

Von diesem Hub führt ein dritter Cat.5e-Kabel in den Kabelkanal der Wand, von wo aus es bis zum Rack im Netzwerkschrank verlegt wird. Im unteren Patch-Panel werden die ankommenden Cat.5e-Netzwerkkabel vom linken PC Port 10 an den Eingang 25 und den rechten PC Port 12 an den Eingang 24 im oben montierten Ethernet-Switch verteilt. Dieser Switch ist über ein Lichtwellenleiter-Kabel (LWL-Kabel) mit Port 1 des obersten Patch-Panel (KVI/SKP/R2.4.22) verbunden. Die Ports sind in \ref{tb:Phys. Netzwerk} gelistet.

Dieses oberste Patch-Panel verfügt über einen einzigen Ausgang. Von dort führt ein LWL-Kabel weiter in den nächsten Raum.

\begin{table}[h!]
\caption{Auflistung der Netzwerkverbindungen der Computer am Arbeitsplatz AG-12 Raum 2.4.22}\label{tb:Phys. Netzwerk}
\centering
\begin{tabular}{|l|l|l|l|}
\hline
Computer 	 & Rack Patch-Panel & Patch-Panel Port  & Switch Port \\ \hline
\computerOne & Feld C 		   	& 12 			 	& 24		    \\ \hline
\computerTwo & Feld C		   	& 10				& 25			\\ \hline
\end{tabular}
\end{table}

Die Kabel folgen einem Farbsystem. Zwischen Netzwerkkarten und dem ersten Patch-Panel sind gelbe Kabel verlegt. Das Kabel zwischen Patch-Panel im Netzwerkschrank und Switch ist für jeden Arbeitsplatz blau für den rechten und rot für den linken Computer.

Nertzwerkschrank Inhalte:
\begin{enumerate}[noitemsep]
\item Blindabdeckungen.
\item Fachböden und Schubladen.
\item Zwei Patchpanele an der Ober- und Unterseite 
\item Ethernet-Switch
\end{enumerate}

\subsubsection{Netzwerkkomponenten des Arbeitsplatzes}\label{Netzwerkomponenten}

Aktive Netzwerkkomponenten:
\begin{itemize}[noitemsep]
\item Netzwerkkarte
\item Ethernet Hub
\item Switch
\end{itemize}

Passive Netzwerkkomponenten:
\begin{itemize}[noitemsep]
\item Patchkabel
\item Patchfelder
\item Netzwerk-Kabel
\item Netzwerkschränke
\end{itemize}


