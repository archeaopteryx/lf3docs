\section{Netzwerkverbindungen}\label{sec:Netzwerk}


\subsection{Alle Netzwerkverbindungen und Netzwerkkomponenten des Arbeitsplatzes }\label{Dokumentation}

Ein Cat.5e-Netzwerkkabel ist vom Computer an die Netzwerkkarte 2 (Network Interface Card) im Rack
angeschlossen. 

Zwei Cat.5e-Netzwerkkabel Verbindungen werden im Rack weitergeführt: 

\begin{enumerate}
\item Vom Netzwerkkarte 2 läuft ein gelber Cat5e ins Ethernet 12C
\item Vom Computer 1 läuft ein blauer Cat.5e-Netzwerkkabel in Konsole Switch 1
\end{enumerate}

Dann geht ein Cat.5e Netzwerkkabel vom Rack in den Netzwerkschrank.

Was gehört in den Nertzwerkschrank

\begin{enumerate}
\item Blindabdeckungen.
\item Fachböden und Schubladen.
\item Zwei Patchpanele an der Ober- und Unterseite 
\item Ethernet-Switch
\end{enumerate}

Im unteren Patch Panel werden die einkommenden Cat.5e-Netzwerkkabel von dem linken PC Port 10 an den Eingang 25 und dem rechten PC Port 12 an den Eingang 24 im oben montierten Ethernet-Switch
verteilt. Die Kabel im Switch werden im oberesten Patch Panel über einen LWL Patchkabel im Port 1 eingeführt.

Dieses oberste Patch Panel verfügt über einen einzigen Ausgang, von dort geht ein Lichtwellenleiter Kabel weiter in den nächsten Raum.

\subsubsection{ Netzwerkkomponenten des Arbeitsplatzes}\label{Netzwerkomponenten}

Aktive Netzwerkkomponenten

Netzwerkkarte
Ethernet Hub
Switch

Passive Netzwerkkomponenten

Patchkabel
Patchfelder
Netzwerk-Kabel
Netzwerkschränke

